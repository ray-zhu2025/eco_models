Input to the Precipitation (PCP) Package is read from the file that has type ``PCP6'' in the Name File.  Any number of PCP Packages can be specified for a single surface water flow model.

\vspace{5mm}
\subsubsection{Structure of Blocks}
\vspace{5mm}

\noindent \textit{FOR EACH SIMULATION}
\lstinputlisting[style=blockdefinition]{./mf6ivar/tex/olf-pcp-options.dat}
\lstinputlisting[style=blockdefinition]{./mf6ivar/tex/olf-pcp-dimensions.dat}
\vspace{5mm}
\noindent \textit{FOR ANY STRESS PERIOD}
\lstinputlisting[style=blockdefinition]{./mf6ivar/tex/olf-pcp-period.dat}
\packageperioddescription

\vspace{5mm}
\subsubsection{Explanation of Variables}
\begin{description}
\input{./mf6ivar/tex/olf-pcp-desc.tex}
\end{description}

\vspace{5mm}
\subsubsection{Example Input File}
\lstinputlisting[style=inputfile]{./mf6ivar/examples/olf-pcp-example.dat}

%\vspace{5mm}
%\subsubsection{Available observation types}
%Well Package observations include the simulated well rates (\texttt{wel}), the well discharge that is available for the MVR package (\texttt{to-mvr}), and the reduction in the specified \texttt{q} when the \texttt{AUTO\_FLOW\_REDUCE} option is enabled. The data required for each WEL Package observation type is defined in table~\ref{table:gwf-welobstype}. The sum of \texttt{wel} and \texttt{to-mvr} is equal to the simulated well discharge rate, which may be less than the specified \texttt{q} if the \texttt{AUTO\_FLOW\_REDUCE} option is enabled. The \texttt{DNODATA} value is returned if the \texttt{wel-reduction} observation is specified but the \texttt{AUTO\_FLOW\_REDUCE} option is not enabled. Negative and positive values for an observation represent a loss from and gain to the GWF model, respectively.

%\begin{longtable}{p{2cm} p{2.75cm} p{2cm} p{1.25cm} p{7cm}}
%\caption{Available WEL Package observation types} \tabularnewline

%\hline
%\hline
%\textbf{Stress Package} & \textbf{Observation type} & \textbf{ID} & \textbf{ID2} & \textbf{Description} \\
%\hline
%\endhead

%\hline
%\endfoot

%WEL & wel & cellid or boundname & -- & Flow between the groundwater system and a well boundary or a group of well boundaries. \\
WEL & to-mvr & cellid or boundname & -- & Well boundary discharge that is available for the MVR package for a well boundary or a group of well boundaries. \\
WEL & wel-reduction & cellid or boundname & -- & Reduction in the specified well boundary discharge calculated when the \texttt{AUTO\_FLOW\_REDUCE} option is specified.
%\label{table:gwf-welobstype}
%\end{longtable}

%\vspace{5mm}
%\subsubsection{Example Observation Input File}
%\lstinputlisting[style=inputfile]{./mf6ivar/examples/gwf-wel-example-obs.dat}
