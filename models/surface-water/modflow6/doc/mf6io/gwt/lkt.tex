Lake Transport (LKT) Package information is read from the file that is specified by ``LKT6'' as the file type.  There can be as many LKT Packages as necessary for a GWT model. Each LKT Package is designed to work with flows from a single corresponding GWF LAK Package. By default \mf uses the LKT package name to determine which LAK Package corresponds to the LKT Package.  Therefore, the package name of the LKT Package (as specified in the GWT name file) must match with the name of the corresponding LAK Package (as specified in the GWF name file).  Alternatively, the name of the flow package can be specified using the FLOW\_PACKAGE\_NAME keyword in the options block.  The GWT LKT Package cannot be used without a corresponding GWF LAK Package.

The LKT Package does not have a dimensions block; instead, dimensions for the LKT Package are set using the dimensions from the corresponding LAK Package.  For example, the LAK Package requires specification of the number of lakes (NLAKES).  LKT sets the number of lakes equal to NLAKES.  Therefore, the PACKAGEDATA block below must have NLAKES entries in it.

\vspace{5mm}
\subsubsection{Structure of Blocks}
\lstinputlisting[style=blockdefinition]{./mf6ivar/tex/gwt-lkt-options.dat}
\lstinputlisting[style=blockdefinition]{./mf6ivar/tex/gwt-lkt-packagedata.dat}
\lstinputlisting[style=blockdefinition]{./mf6ivar/tex/gwt-lkt-period.dat}

\vspace{5mm}
\subsubsection{Explanation of Variables}
\begin{description}
\input{./mf6ivar/tex/gwt-lkt-desc.tex}
\end{description}

\vspace{5mm}
\subsubsection{Example Input File}
\lstinputlisting[style=inputfile]{./mf6ivar/examples/gwt-lkt-example.dat}

\vspace{5mm}
\subsubsection{Available observation types}
Lake Transport Package observations include lake concentration and all of the terms that contribute to the continuity equation for each lake. Additional LKT Package observations include mass flow rates for individual outlets, lakes, or groups of lakes (\texttt{outlet}). The data required for each LKT Package observation type is defined in table~\ref{table:gwt-lktobstype}. Negative and positive values for \texttt{lkt} observations represent a loss from and gain to the GWT model, respectively. For all other flow terms, negative and positive values represent a loss from and gain from the LKT package, respectively.

\begin{longtable}{p{2cm} p{2.75cm} p{2cm} p{1.25cm} p{7cm}}
\caption{Available LKT Package observation types} \tabularnewline

\hline
\hline
\textbf{Stress Package} & \textbf{Observation type} & \textbf{ID} & \textbf{ID2} & \textbf{Description} \\
\hline
\endfirsthead

\captionsetup{textformat=simple}
\caption*{\textbf{Table \arabic{table}.}{\quad}Available LKT Package observation types.---Continued} \tabularnewline

\hline
\hline
\textbf{Stress Package} & \textbf{Observation type} & \textbf{ID} & \textbf{ID2} & \textbf{Description} \\
\hline
\endhead


\hline
\endfoot

% general APT observations
LKT & concentration & ifno or boundname & -- & Lake concentration. If boundname is specified, boundname must be unique for each lake. \\
LKT & flow-ja-face & ifno or boundname & ifno or -- & Mass flow between two lakes connected by an outlet.  If more than one outlet is used to connect the same two lakes, then the mass flow for only the first outlet can be observed.  If a boundname is specified for ID1, then the result is the total mass flow for all outlets for a lake. If a boundname is specified for ID1 then ID2 is not used.\\
LKT & storage & ifno or boundname & -- & Simulated mass storage flow rate for a lake or group of lakes. \\
LKT & constant & ifno or boundname & -- & Simulated mass constant-flow rate for a lake or group of lakes. \\
LKT & from-mvr & ifno or boundname & -- & Simulated mass inflow into a lake or group of lakes from the MVT package. Mass inflow is calculated as the product of provider concentration and the mover flow rate. \\
LKT & to-mvr & outletno or boundname & -- & Mass outflow from a lake outlet, a lake, or a group of lakes that is available for the MVR package. If boundname is not specified for ID, then the outflow available for the MVR package from a specific lake outlet is observed. In this case, ID is the outlet number, which must be between 1 and NOUTLETS. \\
LKT & lkt & ifno or boundname & \texttt{iconn} or -- & Mass flow rate for a lake or group of lakes and its aquifer connection(s). If boundname is not specified for ID, then the simulated lake-aquifer flow rate at a specific lake connection is observed. In this case, ID2 must be specified and is the connection number \texttt{iconn} for lake \texttt{ifno}. \\

%observations specific to the lake package
% rainfall evaporation runoff ext-inflow withdrawal outflow
LKT & rainfall & ifno or boundname & -- & Rainfall rate applied to a lake or group of lakes multiplied by the rainfall concentration. \\
LKT & evaporation & ifno or boundname & -- & Simulated evaporation rate from a lake or group of lakes multiplied by the evaporation concentration. \\
LKT & runoff & ifno or boundname & -- & Runoff rate applied to a lake or group of lakes multiplied by the runoff concentration. \\
LKT & ext-inflow & ifno or boundname & -- & Mass inflow into a lake or group of lakes calculated as the external inflow rate multiplied by the inflow concentration. \\
LKT & withdrawal & ifno or boundname & -- & Specified withdrawal rate from a lake or group of lakes multiplied by the simulated lake concentration. \\
LKT & ext-outflow & ifno or boundname & -- & External outflow from a lake or a group of lakes, through their outlets, to an external boundary.  If the water mover is active, the reported ext-outflow value plus the rate to mover is equal to the total outlet outflow.

%LKT & outlet-inflow & ifno or boundname & -- & Simulated inflow from upstream lake outlets into a lake or group of lakes. \\
%LKT & inflow & ifno or boundname & -- & Sum of specified inflow and simulated inflow from upstream lake outlets into a lake or group of lakes. \\
%LKT & outlet & outletno or boundname & -- & Simulate outlet flow rate from a lake outlet, a lake, or a group of lakes. If boundname is not specified for ID, then the flow from a specific lake outlet is observed. In this case, ID is the outlet number outletno. \\
%LKT & volume & ifno or boundname & -- & Simulated lake volume or group of lakes. \\
%LKT & surface-area & ifno or boundname & -- & Simulated surface area for a lake or group of lakes. \\
%LKT & wetted-area & ifno or boundname & \texttt{iconn} or -- & Simulated wetted-area for a lake or group of lakes and its aquifer connection(s). If boundname is not specified for ID, then the wetted area of a specific lake connection is observed. In this case, ID2 must be specified and is the connection number \texttt{iconn}. \\
%LKT & conductance & ifno or boundname & \texttt{iconn} or -- & Calculated conductance for a lake or group of lakes and its aquifer connection(s). If boundname is not specified for ID, then the calculated conductance of a specific lake connection is observed. In this case, ID2 must be specified and is the connection number \texttt{iconn}.

\label{table:gwt-lktobstype}
\end{longtable}

\vspace{5mm}
\subsubsection{Example Observation Input File}
\lstinputlisting[style=inputfile]{./mf6ivar/examples/gwt-lkt-example-obs.dat}


