Streamflow Transport (SFT) Package information is read from the file that is specified by ``SFT6'' as the file type.  There can be as many SFT Packages as necessary for a GWT model. Each SFT Package is designed to work with flows from a corresponding GWF SFR Package. By default \mf uses the SFT package name to determine which SFR Package corresponds to the SFT Package.  Therefore, the package name of the SFT Package (as specified in the GWT name file) must match with the name of the corresponding SFR Package (as specified in the GWF name file).  Alternatively, the name of the flow package can be specified using the FLOW\_PACKAGE\_NAME keyword in the options block.  The GWT SFT Package cannot be used without a corresponding GWF SFR Package.

The SFT Package does not have a dimensions block; instead, dimensions for the SFT Package are set using the dimensions from the corresponding SFR Package.  For example, the SFR Package requires specification of the number of reaches (NREACHES).  SFT sets the number of reaches equal to NREACHES.  Therefore, the PACKAGEDATA block below must have NREACHES entries in it.

\vspace{5mm}
\subsubsection{Structure of Blocks}
\lstinputlisting[style=blockdefinition]{./mf6ivar/tex/gwt-sft-options.dat}
\lstinputlisting[style=blockdefinition]{./mf6ivar/tex/gwt-sft-packagedata.dat}
\lstinputlisting[style=blockdefinition]{./mf6ivar/tex/gwt-sft-period.dat}

\vspace{5mm}
\subsubsection{Explanation of Variables}
\begin{description}
\input{./mf6ivar/tex/gwt-sft-desc.tex}
\end{description}

\vspace{5mm}
\subsubsection{Example Input File}
\lstinputlisting[style=inputfile]{./mf6ivar/examples/gwt-sft-example.dat}

\vspace{5mm}
\subsubsection{Available observation types}
Streamflow Transport Package observations include reach concentration and all of the terms that contribute to the continuity equation for each reach. Additional SFT Package observations include mass flow rates for individual reaches, or groups of reaches. The data required for each SFT Package observation type is defined in table~\ref{table:gwt-sftobstype}. Negative and positive values for \texttt{sft} observations represent a loss from and gain to the GWT model, respectively. For all other flow terms, negative and positive values represent a loss from and gain from the SFT package, respectively.

\begin{longtable}{p{2cm} p{2.75cm} p{2cm} p{1.25cm} p{7cm}}
\caption{Available SFT Package observation types} \tabularnewline

\hline
\hline
\textbf{Stress Package} & \textbf{Observation type} & \textbf{ID} & \textbf{ID2} & \textbf{Description} \\
\hline
\endfirsthead

\captionsetup{textformat=simple}
\caption*{\textbf{Table \arabic{table}.}{\quad}Available SFT Package observation types.---Continued} \tabularnewline

\hline
\hline
\textbf{Stress Package} & \textbf{Observation type} & \textbf{ID} & \textbf{ID2} & \textbf{Description} \\
\hline
\endhead


\hline
\endfoot

% general APT observations
SFT & concentration & ifno or boundname & -- & Reach concentration. If boundname is specified, boundname must be unique for each reach. \\
SFT & flow-ja-face & ifno or boundname & ifno or -- & Mass flow between two reaches.  If a boundname is specified for ID1, then the result is the total mass flow for all reaches. If a boundname is specified for ID1 then ID2 is not used.\\
SFT & storage & ifno or boundname & -- & Simulated mass storage flow rate for a reach or group of reaches. \\
SFT & constant & ifno or boundname & -- & Simulated mass constant-flow rate for a reach or group of reaches. \\
SFT & from-mvr & ifno or boundname & -- & Simulated mass inflow into a reach or group of reaches from the MVT package. Mass inflow is calculated as the product of provider concentration and the mover flow rate. \\
SFT & to-mvr & ifno or boundname & -- & Mass outflow from a reach, or a group of reaches that is available for the MVR package. If boundname is not specified for ID, then the outflow available for the MVR package from a specific reach is observed. \\
SFT & sft & ifno or boundname & -- & Mass flow rate for a reach or group of reaches and its aquifer connection(s). \\

%observations specific to the stream transport package
% rainfall evaporation runoff ext-inflow withdrawal outflow
SFT & rainfall & ifno or boundname & -- & Rainfall rate applied to a reach or group of reaches multiplied by the rainfall concentration. \\
SFT & evaporation & ifno or boundname & -- & Simulated evaporation rate from a reach or group of reaches multiplied by the evaporation concentration. \\
SFT & runoff & ifno or boundname & -- & Runoff rate applied to a reach or group of reaches multiplied by the runoff concentration. \\
SFT & ext-inflow & ifno or boundname & -- & Mass inflow into a reach or group of reaches calculated as the external inflow rate multiplied by the inflow concentration. \\
SFT & ext-outflow & ifno or boundname & -- & External outflow from a reach or group of reaches to an external boundary. If boundname is not specified for ID, then the external outflow from a specific reach is observed. In this case, ID is the reach ifno.

\label{table:gwt-sftobstype}
\end{longtable}

\vspace{5mm}
\subsubsection{Example Observation Input File}
\lstinputlisting[style=inputfile]{./mf6ivar/examples/gwt-sft-example-obs.dat}


