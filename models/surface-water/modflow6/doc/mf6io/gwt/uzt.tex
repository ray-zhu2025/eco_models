Unsaturated Zone Transport (UZT) Package information is read from the file that is specified by ``UZT6'' as the file type.  There can be as many UZT Packages as necessary for a GWT model. Each UZT Package is designed to work with flows from a corresponding GWF UZF Package. By default \mf uses the UZT package name to determine which UZF Package corresponds to the UZT Package.  Therefore, the package name of the UZT Package (as specified in the GWT name file) must match with the name of the corresponding UZF Package (as specified in the GWF name file).  Alternatively, the name of the flow package can be specified using the FLOW\_PACKAGE\_NAME keyword in the options block.  The GWT UZT Package cannot be used without a corresponding GWF UZF Package.

The UZT Package does not have a dimensions block; instead, dimensions for the UZT Package are set using the dimensions from the corresponding UZF Package.  For example, the UZF Package requires specification of the number of cells (NUZFCELLS).  UZT sets the number of UZT cells equal to NUZFCELLS.  Therefore, the PACKAGEDATA block below must have NUZFCELLS entries in it.

\vspace{5mm}
\subsubsection{Structure of Blocks}
\lstinputlisting[style=blockdefinition]{./mf6ivar/tex/gwt-uzt-options.dat}
\lstinputlisting[style=blockdefinition]{./mf6ivar/tex/gwt-uzt-packagedata.dat}
\lstinputlisting[style=blockdefinition]{./mf6ivar/tex/gwt-uzt-period.dat}

\vspace{5mm}
\subsubsection{Explanation of Variables}
\begin{description}
\input{./mf6ivar/tex/gwt-uzt-desc.tex}
\end{description}

\vspace{5mm}
\subsubsection{Example Input File}
\lstinputlisting[style=inputfile]{./mf6ivar/examples/gwt-uzt-example.dat}

\vspace{5mm}
\subsubsection{Available observation types}
Unsaturated Zone Transport Package observations include UZF cell concentration and all of the terms that contribute to the continuity equation for each UZF cell. Additional UZT Package observations include mass flow rates for individual UZF cells, or groups of UZF cells. The data required for each UZT Package observation type is defined in table~\ref{table:gwt-uztobstype}. Negative and positive values for \texttt{uzt} observations represent a loss from and gain to the GWT model, respectively. For all other flow terms, negative and positive values represent a loss from and gain from the UZT package, respectively.

\begin{longtable}{p{2cm} p{2.75cm} p{2cm} p{1.25cm} p{7cm}}
\caption{Available UZT Package observation types} \tabularnewline

\hline
\hline
\textbf{Stress Package} & \textbf{Observation type} & \textbf{ID} & \textbf{ID2} & \textbf{Description} \\
\hline
\endfirsthead

\captionsetup{textformat=simple}
\caption*{\textbf{Table \arabic{table}.}{\quad}Available UZT Package observation types.---Continued} \tabularnewline

\hline
\hline
\textbf{Stress Package} & \textbf{Observation type} & \textbf{ID} & \textbf{ID2} & \textbf{Description} \\
\hline
\endhead


\hline
\endfoot

% general APT observations
UZT & concentration & ifno or boundname & -- & uzt cell concentration. If boundname is specified, boundname must be unique for each uzt cell. \\
UZT & flow-ja-face & ifno or boundname & ifno or -- & Mass flow between two uzt cells.  If a boundname is specified for ID1, then the result is the total mass flow for all uzt cells. If a boundname is specified for ID1 then ID2 is not used.\\
UZT & storage & ifno or boundname & -- & Simulated mass storage flow rate for a uzt cell or group of uzt cells. \\
UZT & constant & ifno or boundname & -- & Simulated mass constant-flow rate for a uzt cell or a group of uzt cells. \\
UZT & from-mvr & ifno or boundname & -- & Simulated mass inflow into a uzt cell or group of uzt cells from the MVT package. Mass inflow is calculated as the product of provider concentration and the mover flow rate. \\
UZT & uzt & ifno or boundname & -- & Mass flow rate for a uzt cell or group of uzt cells and its aquifer connection(s). \\

%observations specific to the uzt package
% infiltration rej-inf uzet rej-inf-to-mvr
UZT & infiltration & ifno or boundname & -- & Infiltration rate applied to a uzt cell or group of uzt cells multiplied by the infiltration concentration. \\
UZT & rej-inf & ifno or boundname & -- & Rejected infiltration rate applied to a uzt cell or group of uzt cells multiplied by the infiltration concentration. \\
UZT & uzet & ifno or boundname & -- & Unsaturated zone evapotranspiration rate applied to a uzt cell or group of uzt cells multiplied by the uzt cell concentration. \\
UZT & rej-inf-to-mvr & ifno or boundname & -- & Rejected infiltration rate applied to a uzt cell or group of uzt cells multiplied by the infiltration concentration that is sent to the mover package. \\

\label{table:gwt-uztobstype}
\end{longtable}

\vspace{5mm}
\subsubsection{Example Observation Input File}
\lstinputlisting[style=inputfile]{./mf6ivar/examples/gwt-uzt-example-obs.dat}


