Unsaturated Zone Energy Transport (UZE) Package information is read from the file that is specified by ``UZE6'' as the file type.  There can be as many UZE Packages as necessary for a GWE model. Each UZE Package is designed to work with flows from a corresponding GWF UZF Package. By default \mf uses the UZE package name to determine which UZF Package corresponds to the UZE Package.  Therefore, the package name of the UZE Package (as specified in the GWE name file) must match with the name of the corresponding UZF Package (as specified in the GWF name file).  Alternatively, the name of the flow package can be specified using the FLOW\_PACKAGE\_NAME keyword in the options block.  The GWE UZE Package cannot be used without a corresponding GWF UZF Package.

The UZE Package does not have a dimensions block; instead, dimensions for the UZE Package are set using the dimensions from the corresponding UZF Package.  For example, the UZF Package requires specification of the number of cells (NUZFCELLS).  UZE sets the number of UZE cells equal to NUZFCELLS.  Therefore, the PACKAGEDATA block below must have NUZFCELLS entries in it.  Furthermore, UZE requires the area of each UZE object to equal to the area of the grid cell hosting the corresponding UZF object.  If the area of a UZF object is different than the host grid cell, the GWE model will exit with an error message indicating which cell is in violation of this condition. This check is unique to UZE as UZT does not require the area of the corresponding UZF object to equal the area of the host grid cell.  If this error condition occurs, users should check whether the AUXMULTNAME option in the OPTIONS block of the corresponding UZF input file is activated.  Users could inadvertently circumvent the error check by creating two (or more) UZF objects in the same cell using multiple UZF input packages that each specify a UZF object for the same cell.

\vspace{5mm}
\subsubsection{Structure of Blocks}
\lstinputlisting[style=blockdefinition]{./mf6ivar/tex/gwe-uze-options.dat}
\lstinputlisting[style=blockdefinition]{./mf6ivar/tex/gwe-uze-packagedata.dat}
\lstinputlisting[style=blockdefinition]{./mf6ivar/tex/gwe-uze-period.dat}

\vspace{5mm}
\subsubsection{Explanation of Variables}
\begin{description}
\input{./mf6ivar/tex/gwe-uze-desc.tex}
\end{description}

\vspace{5mm}
\subsubsection{Example Input File}
\lstinputlisting[style=inputfile]{./mf6ivar/examples/gwe-uze-example.dat}

\vspace{5mm}
\subsubsection{Available observation types}
Unsaturated Zone Energy Transport Package observations include UZF cell temperature and all of the terms that contribute to the continuity equation for each UZE cell. Additional UZE Package observations include energy flow rates for individual UZE cells, or groups of UZE cells. The data required for each UZE Package observation type is defined in table~\ref{table:gwe-uzeobstype}. Negative and positive values for \texttt{uzt} observations represent a loss from and gain to the GWE model, respectively. For all other flow terms, negative and positive values represent a loss from and gain from the UZE package, respectively.

\begin{longtable}{p{2cm} p{2.75cm} p{2cm} p{1.25cm} p{7cm}}
\caption{Available UZE Package observation types} \tabularnewline

\hline
\hline
\textbf{Stress Package} & \textbf{Observation type} & \textbf{ID} & \textbf{ID2} & \textbf{Description} \\
\hline
\endfirsthead

\captionsetup{textformat=simple}
\caption*{\textbf{Table \arabic{table}.}{\quad}Available UZE Package observation types.---Continued} \tabularnewline

\hline
\hline
\textbf{Stress Package} & \textbf{Observation type} & \textbf{ID} & \textbf{ID2} & \textbf{Description} \\
\hline
\endhead


\hline
\endfoot

% general APT observations
UZE & temperature & uzeno or boundname & -- & uze cell temperature. If boundname is specified, boundname must be unique for each uze cell. \\
UZE & flow-ja-face & uzeno or boundname & uzeno or -- & Energy flow between two uze cells.  If a boundname is specified for ID1, then the result is the total energy flow for all uze cells. If a boundname is specified for ID1 then ID2 is not used.\\
UZE & storage & uzeno or boundname & -- & Simulated energy storage flow rate for a uze cell or group of uze cells. \\
UZE & constant & uzeno or boundname & -- & Simulated energy constant-flow rate for a uze cell or a group of uze cells. \\
UZE & from-mvr & uzeno or boundname & -- & Simulated energy inflow into a uze cell or group of uze cells from the MVE package. Energy inflow is calculated as the product of provider temperature and the mover flow rate. \\
UZE & uze & uzeno or boundname & -- & Energy flow rate for a uze cell or group of uze cells and its aquifer connection(s). \\

%observations specific to the uze package
% infiltration rej-inf uzet rej-inf-to-mvr
UZE & infiltration & uzeno or boundname & -- & Infiltration rate applied to a uze cell or group of uze cells multiplied by the infiltration temperature. \\
UZE & rej-inf & uzeno or boundname & -- & Rejected infiltration rate applied to a uze cell or group of uze cells multiplied by the infiltration temperature. \\
UZE & uzet & uzeno or boundname & -- & Unsaturated zone evapotranspiration rate applied to a uze cell or group of uze cells multiplied by the uze cell temperature. \\
UZE & rej-inf-to-mvr & uzeno or boundname & -- & Rejected infiltration rate applied to a uze cell or group of uze cells multiplied by the infiltration temperature that is sent to the mover package. \\

\label{table:gwe-uzeobstype}
\end{longtable}

\vspace{5mm}
\subsubsection{Example Observation Input File}
\lstinputlisting[style=inputfile]{./mf6ivar/examples/gwe-uze-example-obs.dat}


