Streamflow Energy Transport (SFE) Package information is read from the file that is specified by ``SFE6'' as the file type.  There can be as many SFE Packages as necessary for a GWE model. Each SFE Package is designed to work with flows from a corresponding GWF SFR Package. By default \mf uses the SFE package name to determine which SFR Package corresponds to the SFE Package.  Therefore, the package name of the SFE Package (as specified in the GWE name file) must match with the name of the corresponding SFR Package (as specified in the GWF name file).  Alternatively, the name of the flow package can be specified using the FLOW\_PACKAGE\_NAME keyword in the options block.  The GWE SFE Package cannot be used without a corresponding GWF SFR Package.

The SFE Package does not have a dimensions block; instead, dimensions for the SFE Package are set using the dimensions from the corresponding SFR Package.  For example, the SFR Package requires specification of the number of reaches (NREACHES).  SFE sets the number of reaches equal to NREACHES.  Therefore, the PACKAGEDATA block below must have NREACHES entries in it.

\vspace{5mm}
\subsubsection{Structure of Blocks}
\lstinputlisting[style=blockdefinition]{./mf6ivar/tex/gwe-sfe-options.dat}
\lstinputlisting[style=blockdefinition]{./mf6ivar/tex/gwe-sfe-packagedata.dat}
\lstinputlisting[style=blockdefinition]{./mf6ivar/tex/gwe-sfe-period.dat}

\vspace{5mm}
\subsubsection{Explanation of Variables}
\begin{description}
\input{./mf6ivar/tex/gwe-sfe-desc.tex}
\end{description}

\vspace{5mm}
\subsubsection{Example Input File}
\lstinputlisting[style=inputfile]{./mf6ivar/examples/gwe-sfe-example.dat}

\vspace{5mm}
\subsubsection{Available observation types}
Streamflow Energy Transport Package observations include reach temperature and all of the terms that contribute to the continuity equation for each reach. Additional SFE Package observations include energy flow rates for individual reaches, or groups of reaches. The data required for each SFE Package observation type is defined in table~\ref{table:gwe-sfeobstype}. Negative and positive values for \texttt{sfe} observations represent a loss from and gain to the GWE model, respectively. For all other flow terms, negative and positive values represent a loss from and gain from the SFE package, respectively.

\begin{longtable}{p{2cm} p{2.75cm} p{2cm} p{1.25cm} p{7cm}}
\caption{Available SFE Package observation types} \tabularnewline

\hline
\hline
\textbf{Stress Package} & \textbf{Observation type} & \textbf{ID} & \textbf{ID2} & \textbf{Description} \\
\hline
\endfirsthead

\captionsetup{textformat=simple}
\caption*{\textbf{Table \arabic{table}.}{\quad}Available SFE Package observation types.---Continued} \tabularnewline

\hline
\hline
\textbf{Stress Package} & \textbf{Observation type} & \textbf{ID} & \textbf{ID2} & \textbf{Description} \\
\hline
\endhead


\hline
\endfoot

% general APT observations
SFE & temperature & rno or boundname & -- & Reach temperature. If boundname is specified, boundname must be unique for each reach. \\
SFE & flow-ja-face & rno or boundname & rno or -- & Energy flow between two reaches.  If a boundname is specified for ID1, then the result is the total energy flow for all reaches. If a boundname is specified for ID1 then ID2 is not used.\\
SFE & storage & rno or boundname & -- & Simulated energy storage flow rate for a reach or group of reaches. \\
SFE & constant & rno or boundname & -- & Simulated energy constant-flow rate for a reach or group of reaches. \\
SFE & from-mvr & rno or boundname & -- & Simulated energy inflow into a reach or group of reaches from the MVE package. Energy inflow is calculated as the product of provider temperature and the mover flow rate. \\
SFE & to-mvr & rno or boundname & -- & Energy outflow from a reach, or a group of reaches that is available for the MVR package. If boundname is not specified for ID, then the outflow available for the MVR package from a specific reach is observed. \\
SFE & sfe & rno or boundname & -- & Energy flow rate for a reach or group of reaches and its aquifer connection(s). \\

%observations specific to the stream energy transport package
% rainfall evaporation runoff ext-inflow withdrawal outflow
SFE & rainfall & rno or boundname & -- & Rainfall rate applied to a reach or group of reaches multiplied by the rainfall temperature. \\
SFE & evaporation & rno or boundname & -- & Simulated evaporation rate from a reach or group of reaches multiplied by the latent heat of vaporization for determining the amount of energy lost from a reach. \\
SFE & runoff & rno or boundname & -- & Runoff rate applied to a reach or group of reaches multiplied by the runoff temperature. \\
SFE & ext-inflow & rno or boundname & -- & Energy inflow into a reach or group of reaches calculated as the external inflow rate multiplied by the inflow temperature. \\
SFE & ext-outflow & rno or boundname & -- & External outflow from a reach or group of reaches to an external boundary. If boundname is not specified for ID, then the external outflow from a specific reach is observed. In this case, ID is the reach rno. \\
SFE & strmbd-cond & rno or boundname & -- & Amount of heat conductively exchanged with the streambed material. 

\label{table:gwe-sfeobstype}
\end{longtable}

\vspace{5mm}
\subsubsection{Example Observation Input File}
\lstinputlisting[style=inputfile]{./mf6ivar/examples/gwe-sfe-example-obs.dat}


