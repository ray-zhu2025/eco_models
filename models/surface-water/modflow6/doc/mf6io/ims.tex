An iterative model solution (IMS) is specified within the SOLUTIONGROUP block in the simulation name file.  The model solution will solve all of the models that are added to it, as specified in the simulation name file, and will include Numerical Exchanges, if they are present.  The iterative model solution requires specification of both nonlinear and linear settings.  

\vspace{5mm}
\subsection{Structure of Blocks}
\lstinputlisting[style=blockdefinition]{./mf6ivar/tex/sln-ims-options.dat}
\lstinputlisting[style=blockdefinition]{./mf6ivar/tex/sln-ims-nonlinear.dat}
\lstinputlisting[style=blockdefinition]{./mf6ivar/tex/sln-ims-linear.dat}

\vspace{5mm}
\subsection{Explanation of Variables}
\begin{description}
\input{./mf6ivar/tex/sln-ims-desc.tex}
\end{description}

\subsection{IMS Default and Specified Complexity Values}

The values that are assigned to the \texttt{nonlinear} and \texttt{linear} variables for the the \texttt{simple}, \texttt{moderate}, and \texttt{complex} complexity options are summarized in table~\ref{table:imsopt}. The values defined for the \texttt{simple} complexity option are assigned if the \texttt{COMPLEXITY} keyword is not specified in the \texttt{OPTIONS} block.

\begin{table}[H]
	\caption{IMS variable values for the available complexity options.}
	\begin{tabular}{| l | c | c | c | }
\hline
\hline
Nonlinear Variable & default/\texttt{simple} & \texttt{moderate} & \texttt{complex} \\
\hline
\texttt{OUTER\_DVCLOSE} & 0.001 & 0.01 & 0.1 \\
\texttt{OUTER\_MAXIMUM} & 25 & 50 & 100 \\
\texttt{UNDER\_RELAXATION} & \texttt{NONE} & \texttt{DBD} & \texttt{DBD} \\
\texttt{UNDER\_RELAXATION\_THETA} & 1.0 & 0.9 & 0.8 \\
\texttt{UNDER\_RELAXATION\_KAPPA} & 0.0 & 0.0001 & 0.0001 \\
\texttt{UNDER\_RELAXATION\_GAMMA} & 0.0 & 0.0 & 0.0 \\
\texttt{UNDER\_RELAXATION\_MOMENTUM} & 0.0 & 0.0 & 0.0 \\
\texttt{BACKTRACKING\_NUMBER} & 0 & 0 & 20 \\
\texttt{BACKTRACKING\_TOLERANCE} & 0.0 & 0.0 & 1.05 \\
\texttt{BACKTRACKING\_REDUCTION\_FACTOR} & 0.0 & 0.0 & 0.1 \\
\texttt{BACKTRACKING\_RESIDUAL\_LIMIT} & 0.0 & 0.0 & 0.002 \\
%\texttt{LINEAR\_SOLVER} & \texttt{LINEAR} & \texttt{LINEAR} & \texttt{LINEAR} \\
\hline
\hline

\hline
\hline
Linear Variable & default/\texttt{simple} & \texttt{moderate} & \texttt{complex} \\
\hline
\texttt{INNER\_MAXIMUM} & 50 & 100 & 500 \\
\texttt{INNER\_DVCLOSE} & 0.001 & 0.01 & 0.1 \\
\texttt{INNER\_RCLOSE} & 0.1 & 0.1 & 0.1 \\
\texttt{RCLOSE\_OPTION} & \texttt{infinity-norm} & \texttt{infinity-norm} & \texttt{infinity-norm} \\
\texttt{LINEAR\_ACCELERATION} & \texttt{CG} & \texttt{BICGSTAB} & \texttt{BICGSTAB} \\
\texttt{RELAXATION\_FACTOR} & 0.0 & 0.97 & 0.0 \\
\texttt{PRECONDITIONER\_LEVELS} & 0 & 0 & 5 \\
\texttt{PRECONDITIONER\_DROP\_TOLERANCE} & 0.0 & 0.0 & 0.0001 \\
\texttt{NUMBER\_ORTHOGONALIZATIONS} & 0 & 0 & 2 \\
\texttt{SCALING\_METHOD} & \texttt{NONE} & \texttt{NONE} & \texttt{NONE} \\
\texttt{REORDERING\_METHOD} & \texttt{NONE} & \texttt{NONE} & \texttt{NONE} \\
\hline
\hline

%\hline
%\hline
%$\chi$MD Variable & default/\texttt{simple} & \texttt{moderate} & \texttt{complex} \\
%\hline
%\texttt{INNER\_MAXIMUM} & 50 & 100 & 500 \\
%\texttt{INNER\_DVCLOSE} & 0.001 & 0.01 & 0.1 \\
%\texttt{INNER\_RCLOSE} & 0.0 & 0.0 & 0.0 \\
%\texttt{LINEAR\_ACCELERATION} & \texttt{CG} & \texttt{BICGSTAB} & \texttt{BICGSTAB} \\
%\texttt{PRECONDITIONER\_LEVELS} & 3 & 3 & 5 \\
%\texttt{PRECONDITIONER\_DROP\_TOLERANCE} & 0.0 & 0.001 & 0.00001 \\
%\texttt{NUMBER\_ORTHOGONALIZATIONS} & 5 & 5 & 7 \\
%\texttt{RED\_BLACK\_ORDERING} & \texttt{false} & \texttt{false} & \texttt{false} \\
%\texttt{REORDERING\_METHOD} & \texttt{NONE} & \texttt{NONE} & \texttt{RKM} \\
%\hline
%\hline

\end{tabular}
	\label{table:imsopt}
\end{table}

\vspace{5mm}
\subsection{Example Input File}
\lstinputlisting[style=inputfile]{./mf6ivar/examples/sln-ims-example.dat}

