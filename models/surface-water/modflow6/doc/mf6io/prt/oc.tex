Input to the Output Control Option of the Particle Tracking Model is read from the file that is specified as type ``OC6'' in the Name File. If no ``OC6'' file is specified, default output control is used. The Output Control Option determines how and when particle mass budgets are printed to the listing file and/or written to a separate binary output file.  Under the default settings, the particle mass budget is written to the Listing File at the end of every stress period.  The particle mass budget is also written to the list file if the simulation terminates prematurely due to failed convergence.

Output Control data must be specified using words.  The numeric codes supported in earlier MODFLOW versions can no longer be used.

For the PRINT and SAVE options, there is no option to specify individual layers.  Whenever the budget array is printed or saved, all layers are printed or saved.

\vspace{5mm}
\subsubsection{Tracking Events}

The PRT Model distinguishes a number of tracking events that can trigger writing of particle track information to binary and/or CSV output files. When a particle track record is written to an output file, the record includes an IREASON code that indicates the type of event that triggered the output. IREASON codes and their associated tracking events are enumerated in the Particle Track Output subsection of the PRT Model Input and Output section.
 
The Output Control package provides keyword options that correspond to the various types of tracking events that can trigger output. The keywords are of the form TRACK\_\textit{event}, where \textit{event} indicates the type of event. If no tracking event keywords are provided, output is triggered by all types of tracking events. If any tracking event keywords are provided, only the selected types of events trigger output. This mechanism can be used to filter out events that are not of interest, thereby limiting the size of output files and reducing the runtime spent on writing output.
 
For instance, to configure PRT output for an endpoint analysis, provide the TRACK\_RELEASE and TRACK\_TERMINATE keywords. To configure PRT output for a timeseries analysis, provide the TRACK\_USERTIME keyword as well as a tracking time selection via the TRACKTIMES block.

\vspace{5mm}
\subsubsection{Structure of Blocks}
\vspace{5mm}

\noindent \textit{FOR EACH SIMULATION}
\lstinputlisting[style=blockdefinition]{./mf6ivar/tex/prt-oc-options.dat}
\lstinputlisting[style=blockdefinition]{./mf6ivar/tex/prt-oc-dimensions.dat}
\lstinputlisting[style=blockdefinition]{./mf6ivar/tex/prt-oc-tracktimes.dat}
\vspace{5mm}
\noindent \textit{FOR ANY STRESS PERIOD}
\lstinputlisting[style=blockdefinition]{./mf6ivar/tex/prt-oc-period.dat}

\vspace{5mm}
\subsubsection{Explanation of Variables}
\begin{description}
\input{./mf6ivar/tex/prt-oc-desc.tex}
\end{description}

\vspace{5mm}
\subsubsection{Example Input File}
\lstinputlisting[style=inputfile]{./mf6ivar/examples/prt-oc-example.dat}
