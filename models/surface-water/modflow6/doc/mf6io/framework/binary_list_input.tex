\subsubsection{Description of Binary List Input Files}
All floating point variables are written to the binary input files as DOUBLE PRECISION Fortran variables. Integer variables are written to the input files as Fortran integer variables. Auxiliary variables can be included in binary list input files but as indicated previously binary list input files can not be used for packages that include BOUNDNAMES keyword in the OPTIONS block. The format of binary list data are described below.

\vspace{5mm}
\noindent Record 1: \texttt{(CELLID(N),(RLIST(I,N),I=1,NDAT)(AUXVAR(I,N),I=1, NAUX), N=1,NLIST)}\\

\noindent where

\begin{description} \itemsep0pt \parskip0pt \parsep0pt
\item \texttt{CELLID} is the cell identifier, and depends on the type of grid that is used for the simulation.;
\item \texttt{RLIST} is a double precision two-dimensional array of size (NDAT,NLIST) containing the stress package PERIOD data;
\item \texttt{NDAT} is the number of columns in RLIST, which is the number of columns of real data in the stress package PERIOD data;
\item \texttt{AUXVAR} is a double precision two-dimensional array of size (NAUX,NLIST)  containing the auxiliary data for the stress package PERIOD data;
\item \texttt{NAUX} is the number of columns in AUXVAR, which is the number of columns of real auxiliary data the in stress package PERIOD data;
\item \texttt{NLIST} is the size of the list;
\end{description}

\noindent For a structured grid that uses the DIS input file, CELLID is the layer, row, and column. For a grid that uses the DISV input file, CELLID is the layer and CELL2D number. If the model uses the unstructured discretization (DISU) input file, CELLID is the node number for the cell. \texttt{NLIST} must be less than or equal to \texttt{MAXBOUND} for a stress package. \texttt{NAUX} is determined by the number of \texttt{AUXILIARY} variable names define in the OPTIONS block for the stress package.
