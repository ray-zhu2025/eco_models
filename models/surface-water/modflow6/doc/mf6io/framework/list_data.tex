\subsection{List Input}
Some items consist of several variables, such as layer, row, column, stage, and conductance, for example.  List input refers to a block of data with a separate item on each line.  For some common list types, the first set of variables is a cell identifier (denoted as \texttt{cellid} in this guide), such as layer, row, and column. With lists, the input data for each item must start on a new line. All variables for an item are assumed to be contained in a single line.  Each input variable has a data type, which can be Double Precision, Integer, or Character. Integers are whole numbers and must not include a decimal point or exponent. Double Precision numbers can include a decimal point and an exponent. If no decimal point is included in the entered value, then the decimal point is assumed to be at the right side of the value. Any printable character is allowed for character variables. 

Variables starting with the letters I-N are most commonly integers; however, in some instances, a character string may start with the letters I-N. Variables starting with the letters A-H and O-Z are primarily double precision numbers; however, these variable names may also be used for character data.  In \mf all variables are explicitly declared within the source code, as opposed to the implicit type declaration in previous MODFLOW versions.  This explicit declaration means that the variable type can be easily determined from the source code.

Free formatting is used throughout the input instructions.  With free format, values are not required to occupy a fixed number of columns in a line. Each value can occupy one or more columns as required to represent the value; however, the values must still be included in the prescribed order. One or more spaces, or a single comma optionally combined with spaces, must separate adjacent values. Also, a numeric value of zero must be explicitly represented with 0 and not by one or more spaces when free format is used, because detecting the difference between a space that represents 0 and a space that represents a value separator is not possible. Free format is similar to Fortran's list directed input.

Two capabilities included in Fortran's list-directed input are not included in the free-format input implemented in \mf. Null values in which input values are left unchanged from their previous values are not allowed. In general, MODFLOW's input values are not defined prior to their input.  A ``/'' cannot be used to terminate an input line without including values for all the variables; data values for all required input variables must be explicitly specified on an input line.  For character data, MODFLOW's free format implementation is less stringent than the list-directed input of Fortran. Fortran requires character data to be delineated by apostrophes. MODFLOW does not require apostrophes unless a blank or a comma is part of a character variable.

As an example of a list, consider the PERIOD block for the GHB Package.  The input format is  shown below:

\lstinputlisting[style=blockdefinition]{./mf6ivar/tex/gwf-ghb-period.dat}

Each line represents a separate item, which consists of variables.  In this case, the first variable of the item, \texttt{cellid} is an array of size \texttt{ncelldim}.  The next two variables of the item are \texttt{bhead} and \texttt{cond}.  Lastly, the item has two optional variables, \texttt{aux} and \texttt{boundname}.  Three of the variables shown in the list are colored in blue.  Variables that are colored in blue mean that they can be represented with a time series.  The time series capability is described in the section on Time-Variable Input in this document.  

The following is simple example of a PERIOD block for the GHB Package, which shows how a list is entered by the user.

\begin{lstlisting}[style=inputfile]
BEGIN PERIOD 1
#      lay       row       col     stage      cond
         1        13         1     988.0     0.038
         1        14         9    1045.0     0.038
END PERIOD
\end{lstlisting}

As described earlier in the section on ``Block and Keyword Input,'' block information can be read from a separate text file.  To activate reading a list from separate text file, the first and only entry in the block must be a control line of the following form:  

\begin{lstlisting}[style=blockdefinition]
  OPEN/CLOSE <fname>
\end{lstlisting}

\noindent where \texttt{fname} is the name of the file containing the list.  Lists for the stress packages (CHD, WEL, DRN, RIV, GHB, RCH, and EVT) have an additional BINARY option.  The BINARY  option is not supported for the advanced stress packages (LAK, MAW, SFR, UZF, LKT, MWT, SFT, UZT).  The BINARY options is specified as follows:

\begin{lstlisting}[style=blockdefinition]
  OPEN/CLOSE <fname> [(BINARY)]
\end{lstlisting}

If the (BINARY) keyword is found on the control line, then the file is opened as an unformatted file on unit 99, and the list is read.  There are a number of requirements for using the (BINARY) option for lists.  All stress package lists begin with integer values for the \texttt{cellid} (layer, row, and column, for example).  These values must be represented as integer numbers in the unformatted file.  Also, all auxiliary data must be included in the binary file; auxiliary data must be represented as double precision numbers.  Lastly, the (BINARY) option does not support entry of \texttt{boundname}, and so the BOUNDNAMES option should not be activated in the OPTIONS block for the package.  