The simulation name file contains information about simulation options, simulation timing, models that are present in the simulation, how models exchange information, and how models are solved.

The present version of \mf uses the concept of a solution group.  For most simulations, a solution group will contain one solution and one model within that solution.  The solution group is designed, however, so that multiple solutions can be solved together in a picard iteration loop.  This might be used in the future to iteratively couple models that cannot be tightly coupled at the matrix level within a single numerical solution.  The solution group is flexible so that multiple solution groups can be included in a simulation.  More information on solution groups will be added to this document as new model types and exchanges are added that can take advantage of the concept.

The simulation name file is read from a file in the current working directory with the name ``mfsim.nam''.  Input within the simulation name file is provided through the following input blocks, which must be listed in the order shown below.  The options block itself is optional.  All other blocks are required.

\vspace{5mm}
\subsection{Structure of Blocks}
\lstinputlisting[style=blockdefinition]{./mf6ivar/tex/sim-nam-options.dat}
\lstinputlisting[style=blockdefinition]{./mf6ivar/tex/sim-nam-timing.dat}
\lstinputlisting[style=blockdefinition]{./mf6ivar/tex/sim-nam-models.dat}
\lstinputlisting[style=blockdefinition]{./mf6ivar/tex/sim-nam-exchanges.dat}
\lstinputlisting[style=blockdefinition]{./mf6ivar/tex/sim-nam-solutiongroup.dat}

\vspace{5mm}
\subsection{Explanation of Variables}
\begin{description}
\input{./mf6ivar/tex/sim-nam-desc.tex}
\end{description}

\begin{table}[h]
\caption{Model types available in Version \modflowversion}
\small
\begin{center}
\begin{tabular*}{\columnwidth}{l l}
\hline
\hline
Mtype & Type of Model \\
\hline
GWF6 & Groundwater Flow Model \\
GWT6 & Groundwater Transport Model \\
GWE6 & Groundwater Energy Model \\
PRT6 & Particle Tracking Model \\
\hline 
\end{tabular*}
\label{table:mtype}
\end{center}
\normalsize
\end{table}

\begin{table}[h]
\caption{Exchange types available in Version \modflowversion}
\small
\begin{center}
\begin{tabular*}{\columnwidth}{l p{13cm}}
\hline
\hline
Exgtype & Type of Exchange \\
\hline
GWF6-GWF6 & Exchange between two Groundwater Flow Models.  Input for this file is described in a dedicated section in this guide. \\
GWF6-GWT6 & Exchange between a Groundwater Flow Model and a Groundwater Transport Model.  In the present version, a filename is required for this exchange and the file must exist, however, nothing is read from this file.  \\
GWT6-GWT6 & Exchange between two Groundwater Transport Models.  Input for this file is described in a dedicated section in this guide. \\
GWF6-GWE6 & Exchange between a Groundwater Flow Model and a Groundwater Energy Model.  In the present version, a filename is required for this exchange and the file must exist, however, nothing is read from this file.  \\
GWE6-GWE6 & Exchange between two Groundwater Energy Models.  Input for this file is described in a dedicated section in this guide. \\
GWF6-PRT6 & Exchange between a Groundwater Flow Model and a Particle Tracking Model.  In the present version, a filename is required for this exchange and the file must exist, however, nothing is read from this file.  \\
\hline
\end{tabular*}
\label{table:exgtype}
\end{center}
\normalsize
\end{table}

\vspace{5mm}
\subsection{Example Input File}
\lstinputlisting[style=inputfile]{./mf6ivar/examples/sim-nam-example.dat}

