Input to the Evapotranspiration (EVT) Package is read from the file that has type ``EVT6'' in the Name File. Any number of EVT Packages can be specified for a single groundwater flow model. All single-valued variables are free format.

Evapotranspiration input can be specified using lists or arrays.  Array-based input for evapotranspiration is activated by providing READASARRAYS within the OPTIONS block.   Instructions for specifying list-based evapotranspiration is described in the previous section.  Array-based input for evapotranspiration provides a similar approach for providing evapotranspiration rates as previous MODFLOW versions.  Array-based input for evapotranspiration can be used only with the DIS and DISV Packages.  Array-based input for evapotranspiration cannot be used with the DISU Package.

When array-based input is used for evapotranspiration, the DIMENSIONS block should not be specified.  The array size is determined from the model grid.   Segmented evapotranspiration cannot be used with the READASARRAYS option.

\vspace{5mm}
\subsubsection{Structure of Blocks}
\vspace{5mm}

\noindent \textit{FOR EACH SIMULATION}
\lstinputlisting[style=blockdefinition]{./mf6ivar/tex/gwf-evta-options.dat}
\vspace{5mm}
\noindent \textit{FOR ANY STRESS PERIOD}
\lstinputlisting[style=blockdefinition]{./mf6ivar/tex/gwf-evta-period.dat}
\packageperioddescriptionarray{evapotranspiration}

\vspace{5mm}
\subsubsection{Explanation of Variables}
\begin{description}
\input{./mf6ivar/tex/gwf-evta-desc.tex}
\end{description}

\vspace{5mm}
\subsubsection{Example Input File}
\lstinputlisting[style=inputfile]{./mf6ivar/examples/gwf-evta-example.dat}
