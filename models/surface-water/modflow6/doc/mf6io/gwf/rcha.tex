
Input to the Recharge (RCH) Package is read from the file that has type ``RCH6'' in the Name File.  Any number of RCH Packages can be specified for a single groundwater flow model.

Recharge input can be specified using lists or arrays.  Array-based input for recharge is activated by providing READASARRAYS within the OPTIONS block.   Instructions for specifying list-based recharge is described in the previous section.  Array-based input for recharge provides a similar approach for providing recharge rates as previous MODFLOW versions.  Array-based input for recharge can be used only with the DIS and DISV Packages.  Array-based input for recharge cannot be used with the DISU Package.

When array-based input is used for recharge, the DIMENSIONS block should not be specified.  The array size is determined from the model grid. 

\vspace{5mm}
\subsubsection{Structure of Blocks}
\vspace{5mm}

\noindent \textit{FOR EACH SIMULATION}
\lstinputlisting[style=blockdefinition]{./mf6ivar/tex/gwf-rcha-options.dat}
\vspace{5mm}
\noindent \textit{FOR ANY STRESS PERIOD}
\lstinputlisting[style=blockdefinition]{./mf6ivar/tex/gwf-rcha-period.dat}
\packageperioddescriptionarray{recharge}

\vspace{5mm}
\subsubsection{Explanation of Variables}
\begin{description}
\input{./mf6ivar/tex/gwf-rcha-desc.tex}
\end{description}

\vspace{5mm}
\subsubsection{Example Input File}
\lstinputlisting[style=inputfile]{./mf6ivar/examples/gwf-rcha-example.dat}

