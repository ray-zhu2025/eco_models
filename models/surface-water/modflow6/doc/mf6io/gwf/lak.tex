Input to the Lake (LAK) Package is read from the file that has type ``LAK6'' in the Name File.  Any number of LAK Packages can be specified for a single groundwater flow model.

\vspace{5mm}
\subsubsection{Structure of Blocks}
\vspace{5mm}

\noindent \textit{FOR EACH SIMULATION}
\lstinputlisting[style=blockdefinition]{./mf6ivar/tex/gwf-lak-options.dat}
\lstinputlisting[style=blockdefinition]{./mf6ivar/tex/gwf-lak-dimensions.dat}
\lstinputlisting[style=blockdefinition]{./mf6ivar/tex/gwf-lak-packagedata.dat}
\noindent \textit{IF \texttt{nlakeconn} IS GREATER THAN ZERO FOR ANY LAKE}
\lstinputlisting[style=blockdefinition]{./mf6ivar/tex/gwf-lak-connectiondata.dat}
\noindent \textit{IF \texttt{ntables} IS GREATER THAN ZERO}
\lstinputlisting[style=blockdefinition]{./mf6ivar/tex/gwf-lak-tables.dat}
\noindent \textit{IF \texttt{noutlets} IS GREATER THAN ZERO FOR ANY LAKE}
\lstinputlisting[style=blockdefinition]{./mf6ivar/tex/gwf-lak-outlets.dat}

\vspace{5mm}
\noindent \textit{FOR ANY STRESS PERIOD}
\lstinputlisting[style=blockdefinition]{./mf6ivar/tex/gwf-lak-period.dat}
\advancedpackageperioddescription{lake}{lakes}

\vspace{5mm}
\subsubsection{Explanation of Variables}
\begin{description}
\input{./mf6ivar/tex/gwf-lak-desc.tex}
\end{description}

\vspace{5mm}
\subsubsection{Example Input File}
\lstinputlisting[style=inputfile]{./mf6ivar/examples/gwf-lak-example.dat}

\vspace{5mm}
\subsubsection{Available observation types}
Lake Package observations include lake stage and all of the terms that contribute to the continuity equation for each lake. Additional LAK Package observations include flow rates for individual outlets, lakes, or groups of lakes (\texttt{outlet}); the lake volume (\texttt{volume}); lake surface area (\texttt{surface-area}); wetted area for a lake-aquifer connection (\texttt{wetted-area}); and the conductance for a lake-aquifer connection conductance (\texttt{conductance}). The data required for each LAK Package observation type is defined in table~\ref{table:gwf-lakobstype}. Negative and positive values for \texttt{lak} observations represent a loss from and gain to the GWF model, respectively. For all other flow terms, negative and positive values represent a loss from and gain from the LAK package, respectively.

\begin{longtable}{p{2cm} p{2.75cm} p{2cm} p{1.25cm} p{7cm}}
\caption{Available LAK Package observation types} \tabularnewline

\hline
\hline
\textbf{Stress Package} & \textbf{Observation type} & \textbf{ID} & \textbf{ID2} & \textbf{Description} \\
\hline
\endfirsthead

\captionsetup{textformat=simple}
\caption*{\textbf{Table \arabic{table}.}{\quad}Available LAK Package observation types.---Continued} \tabularnewline

\hline
\hline
\textbf{Stress Package} & \textbf{Observation type} & \textbf{ID} & \textbf{ID2} & \textbf{Description} \\
\hline
\endhead


\hline
\endfoot

LAK & stage & ifno or boundname & -- & Surface-water stage in a lake. If boundname is specified, boundname must be unique for each lake. \\
LAK & ext-inflow & ifno or boundname & -- & Specified inflow into a lake or group of lakes. \\
LAK & outlet-inflow & ifno or boundname & -- & Simulated inflow from upstream lake outlets into a lake or group of lakes. \\
LAK & inflow & ifno or boundname & -- & Sum of specified inflow and simulated inflow from upstream lake outlets into a lake or group of lakes. \\
LAK & from-mvr & ifno or boundname & -- & Inflow into a lake or group of lakes from the MVR package. \\
LAK & rainfall & ifno or boundname & -- & Rainfall rate applied to a lake or group of lakes. \\
LAK & runoff & ifno or boundname & -- & Runoff rate applied to a lake or group of lakes. \\
LAK & lak & ifno or boundname & \texttt{iconn} or -- & Simulated flow rate for a lake or group of lakes and its aquifer connection(s). If boundname is not specified for ID, then the simulated lake-aquifer flow rate at a specific lake connection is observed. In this case, ID2 must be specified and is the connection number \texttt{iconn}. \\
LAK & withdrawal & ifno or boundname & -- & Specified withdrawal rate from a lake or group of lakes. \\
LAK & evaporation & ifno or boundname & -- & Simulated evaporation rate from a lake or group of lakes. \\
LAK & ext-outflow & outletno or boundname & -- & External outflow from a lake outlet, a lake, or a group of lakes to an external boundary. If boundname is not specified for ID, then the external outflow from a specific lake outlet is observed. In this case, ID is the outlet number outletno. \\
LAK & to-mvr & outletno or boundname & -- & Outflow from a lake outlet, a lake, or a group of lakes that is available for the MVR package. If boundname is not specified for ID, then the outflow available for the MVR package from a specific lake outlet is observed. In this case, ID is the outlet number outletno. \\
LAK & storage & ifno or boundname & -- & Simulated storage flow rate for a lake or group of lakes. \\
LAK & constant & ifno or boundname & -- & Simulated constant-flow rate for a lake or group of lakes. \\
LAK & outlet & outletno or boundname & -- & Simulated outlet flow rate from a lake outlet, a lake, or a group of lakes. If boundname is not specified for ID, then the flow from a specific lake outlet is observed. In this case, ID is the outlet number outletno. \\
LAK & volume & ifno or boundname & -- & Simulated lake volume or group of lakes. \\
LAK & surface-area & ifno or boundname & -- & Simulated surface area for a lake or group of lakes. \\
LAK & wetted-area & ifno or boundname & \texttt{iconn} or -- & Simulated wetted-area for a lake or group of lakes and its aquifer connection(s). If boundname is not specified for ID, then the wetted area of a specific lake connection is observed. In this case, ID2 must be specified and is the connection number \texttt{iconn}. \\
LAK & conductance & ifno or boundname & \texttt{iconn} or -- & Calculated conductance for a lake or group of lakes and its aquifer connection(s). If boundname is not specified for ID, then the calculated conductance of a specific lake connection is observed. In this case, ID2 must be specified and is the connection number \texttt{iconn}.

\label{table:gwf-lakobstype}
\end{longtable}

\vspace{5mm}
\subsubsection{Example Observation Input File}
\lstinputlisting[style=inputfile]{./mf6ivar/examples/gwf-lak-example-obs.dat}

\newpage
\subsection{Lake Table Input File}
Lake tables of stage, volume, and surface area can be specified for individual lakes.  Lake tables are specified by including file names in the LAKE\_TABLES block of the LAK Package.  These file names correspond to a lake table input file.  The format of the lake table input file is described here.

\vspace{5mm}
\subsubsection{Structure of Blocks}
\vspace{5mm}

\lstinputlisting[style=blockdefinition]{./mf6ivar/tex/utl-laktab-dimensions.dat}
\lstinputlisting[style=blockdefinition]{./mf6ivar/tex/utl-laktab-table.dat}
\vspace{5mm}

\vspace{5mm}
\subsubsection{Explanation of Variables}
\begin{description}
\input{./mf6ivar/tex/utl-laktab-desc.tex}
\end{description}

\subsubsection{Example Input File}
\lstinputlisting[style=inputfile]{./mf6ivar/examples/utl-laktab-example.dat}

