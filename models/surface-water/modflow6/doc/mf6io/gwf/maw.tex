Input to the Multi-Aquifer Well (MAW) Package is read from the file that has type ``MAW6'' in the Name File.  Any number of MAW Packages can be specified for a single groundwater flow model.

\vspace{5mm}
\subsubsection{Structure of Blocks}
\vspace{5mm}

\noindent \textit{FOR EACH SIMULATION}
\lstinputlisting[style=blockdefinition]{./mf6ivar/tex/gwf-maw-options.dat}
\lstinputlisting[style=blockdefinition]{./mf6ivar/tex/gwf-maw-dimensions.dat}
\lstinputlisting[style=blockdefinition]{./mf6ivar/tex/gwf-maw-packagedata.dat}
\lstinputlisting[style=blockdefinition]{./mf6ivar/tex/gwf-maw-connectiondata.dat}
\vspace{5mm}
\noindent \textit{FOR ANY STRESS PERIOD}
\lstinputlisting[style=blockdefinition]{./mf6ivar/tex/gwf-maw-period.dat}
\advancedpackageperioddescription{well}{wells}

\vspace{5mm}
\subsubsection{Explanation of Variables}
\begin{description}
\input{./mf6ivar/tex/gwf-maw-desc.tex}
\end{description}

\vspace{5mm}
\subsubsection{Example Input File -- Conductance Calculated using Thiem Equation}
\lstinputlisting[style=inputfile]{./mf6ivar/examples/gwf-maw-example1.dat}
\subsubsection{Example Input File -- Conductance Calculated using Screen Geometry}
\lstinputlisting[style=inputfile]{./mf6ivar/examples/gwf-maw-example2.dat}
\subsubsection{Example Input File -- Flowing Well with Conductance Specified}
\lstinputlisting[style=inputfile]{./mf6ivar/examples/gwf-maw-example3.dat}

\vspace{5mm}
\subsubsection{Available observation types}
Multi-Aquifer Well Package observations include well head and all of the terms that contribute to the continuity equation for each multi-aquifer well. Additional LAK Package observations include the conductance for a well-aquifer connection conductance (\texttt{conductance}) and the calculated flowing well-aquifer connection conductance (\texttt{fw-conductance}). The data required for each MAW Package observation type is defined in table~\ref{table:gwf-mawobstype}. Negative and positive values for \texttt{maw} observations represent a loss from and gain to the GWF model, respectively. For all other flow terms, negative and positive values represent a loss from and gain from the MAW package, respectively.

\begin{longtable}{p{2cm} p{2.75cm} p{2cm} p{1.25cm} p{7cm}}
\caption{Available MAW Package observation types} \tabularnewline

\hline
\hline
\textbf{Stress Package} & \textbf{Observation type} & \textbf{ID} & \textbf{ID2} & \textbf{Description} \\
\hline
\endfirsthead

\captionsetup{textformat=simple}
\caption*{\textbf{Table \arabic{table}.}{\quad}Available MAW Package observation types.---Continued} \\

\hline
\hline
\textbf{Stress Package} & \textbf{Observation type} & \textbf{ID} & \textbf{ID2} & \textbf{Description} \\
\hline
\endhead

\hline
\endfoot

MAW & head & ifno or boundname & -- & Head in a multi-aquifer well. If boundname is specified, boundname must be unique for each multi-aquifer well. \\
MAW & from-mvr & ifno or boundname & -- & Simulated inflow to a well from the MVR package for a multi-aquifer well or a group of multi-aquifer wells. \\
MAW & maw & ifno or boundname & \texttt{icon} or -- & Simulated flow rate for a multi-aquifer well or a group of multi-aquifer wells and its aquifer connection(s). If boundname is not specified for ID, then the simulated multi-aquifer well-aquifer flow rate at a specific multi-aquifer well connection is observed. In this case, ID2 must be specified and is the connection number \texttt{icon}. \\
MAW & rate & ifno or boundname & -- & Simulated pumping rate for a multi-aquifer well or a group of multi-aquifer wells. \\
MAW & rate-to-mvr & ifno or boundname & -- & Simulated well discharge that is available for the MVR package for a multi-aquifer well or a group of multi-aquifer wells. \\
MAW & fw-rate & ifno or boundname & -- & Simulated flowing well flow rate for a multi-aquifer well or a group of multi-aquifer wells.  \\
MAW & fw-to-mvr & ifno or boundname & -- & Simulated flowing well discharge rate that is available for the MVR package for a multi-aquifer well or a group of multi-aquifer wells. \\
MAW & storage & ifno or boundname & -- & Simulated storage flow rate for a multi-aquifer well or a group of multi-aquifer wells. \\
MAW & constant & ifno or boundname & -- & Simulated constant-flow rate for a multi-aquifer well or a group of multi-aquifer wells. \\
MAW & conductance & ifno or boundname & \texttt{icon} or -- & Simulated well conductance for a multi-aquifer well or a group of multi-aquifer wells and its aquifer connection(s). If boundname is not specified for ID, then the simulated multi-aquifer well conductance at a specific multi-aquifer well connection is observed. In this case, ID2 must be specified and is the connection number \texttt{icon}. \\
MAW & fw-conductance & ifno or boundname & -- & Simulated flowing well conductance for a multi-aquifer well or a group of multi-aquifer wells.
\label{table:gwf-mawobstype}
\end{longtable}

\vspace{5mm}
\subsubsection{Example Observation Input File}
\lstinputlisting[style=inputfile]{./mf6ivar/examples/gwf-maw-example-obs.dat}
