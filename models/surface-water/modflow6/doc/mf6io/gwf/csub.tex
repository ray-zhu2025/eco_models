Input to the Skeletal Storage, Compaction, and Subsidence (CSUB) Package is read from the file that has type ``CSUB6'' in the Name File.  Technical details for the CSUB Package are described in \cite{modflow6csub}.  If the CSUB Package is not included for a model, then storage changes resulting from compaction will not be calculated.  Only one CSUB Package can be specified for a GWF model. Only the first and last stress period can be specified to be STEADY-STATE in the STO Package when the CSUB Package is being used in the GWF model. Also the specific storage (SS) must be specified to be zero in the STO Package for every cell.

\vspace{5mm}
\subsubsection{Structure of Blocks}

\vspace{5mm}
\noindent \textit{FOR EACH SIMULATION}
\lstinputlisting[style=blockdefinition]{./mf6ivar/tex/gwf-csub-options.dat}
\lstinputlisting[style=blockdefinition]{./mf6ivar/tex/gwf-csub-dimensions.dat}
\lstinputlisting[style=blockdefinition]{./mf6ivar/tex/gwf-csub-griddata.dat}
\lstinputlisting[style=blockdefinition]{./mf6ivar/tex/gwf-csub-packagedata.dat}
\vspace{5mm}
\noindent \textit{FOR ANY STRESS PERIOD}
\lstinputlisting[style=blockdefinition]{./mf6ivar/tex/gwf-csub-period.dat}
\packageperioddescription

\vspace{5mm}
\subsubsection{Explanation of Variables}
\begin{description}
\input{./mf6ivar/tex/gwf-csub-desc.tex}
\end{description}

\vspace{5mm}
\subsubsection{Example Input File}
\lstinputlisting[style=inputfile]{./mf6ivar/examples/gwf-csub-example.dat}


\vspace{5mm}
\subsubsection{Available observation types}
Subsidence Package observations include all of the terms that contribute to the continuity equation for each GWF cell. The data required for each CSUB Package observation type is defined in table~\ref{table:gwf-csubobstype}. Negative and positive values for \texttt{CSUB} observations represent a loss from and gain to the GWF model, respectively.


\begin{longtable}{p{2cm} p{2.75cm} p{2cm} p{1.25cm} p{7cm}}
\caption{Available CSUB Package observation types} \tabularnewline

\hline
\hline
\textbf{Stress Package} & \textbf{Observation type} & \textbf{ID} & \textbf{ID2} & \textbf{Description} \\
\hline
\endfirsthead

\captionsetup{textformat=simple}
\caption*{\textbf{Table \arabic{table}.}{\quad}Available CSUB Package observation types.---Continued} \\

\hline
\hline
\textbf{Stress Package} & \textbf{Observation type} & \textbf{ID} & \textbf{ID2} & \textbf{Description} \\
\hline
\endhead

\hline
\multicolumn{5}{l}{\textbf{NOTE}: The NODATA value is reported for steady-state stress periods.} \\
\endfoot

CSUB & csub & icsubno or boundname & -- & Flow between the groundwater system and a interbed or group of interbeds. \\
CSUB & inelastic-csub & icsubno or boundname & -- & Flow between the groundwater system and a interbed or group of interbeds from inelastic compaction. \\
CSUB & elastic-csub & icsubno or boundname & -- & Flow between the groundwater system and a interbed or group of interbeds from elastic compaction. \\
CSUB & coarse-csub & cellid & -- & Flow between the groundwater system and coarse-grained materials in a GWF cell. \\
CSUB & csub-cell & cellid & -- & Flow between the groundwater system for all interbeds and coarse-grained materials in a GWF cell. \\
CSUB & wcomp-csub-cell & cellid & -- & Flow between the groundwater system for all interbeds and coarse-grained materials in a GWF cell from water compressibility. \\

CSUB & sk & icsubno & -- & Convertible interbed storativity in a interbed. Convertible interbed storativity is inelastic interbed storativity if the current effective stress is greater than the preconsolidation stress. \\
CSUB & ske & icsubno & -- & Elastic interbed storativity in a interbed. \\
CSUB & sk-cell & cellid & -- & Convertible interbed and coarse-grained material storativity in a GWF cell. Convertible interbed storativity is inelastic interbed storativity if the current effective stress is greater than the preconsolidation stress. \\
CSUB & ske-cell & cellid & -- & Elastic interbed and coarse-grained material storativity in a GWF cell. \\

CSUB & estress-cell & cellid & -- & effective stress in a GWF cell. \\
CSUB & gstress-cell & cellid & -- & geostatic stress in a GWF cell. \\

CSUB & interbed-compaction & icsubno  & -- & interbed compaction in a interbed. \\
CSUB & interbed-compaction-pct & icsubno  & -- & interbed percent compaction in a interbed. \\
CSUB & inelastic-compaction &  icsubno & -- & inelastic interbed compaction in a interbed. \\
CSUB & elastic-compaction &  icsubno & -- & elastic interbed compaction a interbed. \\
CSUB & coarse-compaction & cellid  & -- & elastic compaction in coarse-grained materials in a GWF cell. \\
CSUB & inelastic-compaction-cell &  cellid & -- & inelastic compaction in all interbeds in a GWF cell. \\
CSUB & elastic-compaction-cell &  cellid & -- & elastic compaction in coarse-grained materials and all interbeds in a GWF cell. \\
CSUB & compaction-cell & cellid  & -- & total compaction in coarse-grained materials and all interbeds in a GWF cell. \\

CSUB & thickness & icsubno & -- & thickness of a interbed. \\
CSUB & coarse-thickness & cellid & -- & thickness of coarse-grained materials in a GWF cell. \\
CSUB & thickness-cell & cellid & -- & total thickness of coarse-grained materials and all interbeds in a GWF cell. \\

CSUB & theta & icsubno & -- & porosity of a interbed. \\
CSUB & coarse-theta & cellid  & -- & porosity of coarse-grained materials in a GWF cell. \\
CSUB & theta-cell & cellid  & -- & thickness-weighted porosity of coarse-grained materials and all interbeds in a GWF cell. \\

CSUB & delay-flowtop & icsubno or boundname  & -- & Flow between the groundwater system and a delay interbed or group of interbeds across the top of the interbed(s). \\
CSUB & delay-flowbot & icsubno or boundname  & -- & Flow between the groundwater system and a delay interbed or group of interbeds across the bottom of the interbed(s). \\

CSUB & delay-head & icsubno & idcellno & head in interbed in delay cell idcellno (1 $<=$ idcellno $<=$ NDELAYCELLS). \\
CSUB & delay-gstress & icsubno  & idcellno & geostatic stress in interbed in delay cell idcellno (1 $<=$ idcellno $<=$ NDELAYCELLS). \\
CSUB & delay-estress & icsubno  & idcellno & effective stress in interbed in delay cell idcellno (1 $<=$ idcellno $<=$ NDELAYCELLS). \\
CSUB & delay-preconstress & icsubno  & idcellno & preconsolidation stress in interbed in delay cell idcellno (1 $<=$ idcellno $<=$ NDELAYCELLS). \\
CSUB & delay-compaction & icsubno  & idcellno & compaction in interbed in delay cell idcellno (1 $<=$ idcellno $<=$ NDELAYCELLS). \\
CSUB & delay-thickness & icsubno  & idcellno & thickness of interbed or group of interbeds in delay cell idcellno (1 $<=$ idcellno $<=$ NDELAYCELLS). \\
CSUB & delay-theta & icsubno  & idcellno & porosity of interbed in delay cell idcellno (1 $<=$ idcellno $<=$ NDELAYCELLS). \\

CSUB & preconstress-cell & cellid  & -- & preconsolidation stress in a GWF cell containing at least one interbed.

\label{table:gwf-csubobstype}
\end{longtable}

\vspace{5mm}
\subsubsection{Example Observation Input File}
\lstinputlisting[style=inputfile]{./mf6ivar/examples/gwf-csub-example-obs.dat}