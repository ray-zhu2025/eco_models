Input to the General-Head Boundary (GHB) Package is read from the file that has type ``GHB6'' in the Name File.  Any number of GHB Packages can be specified for a single groundwater flow model.

\vspace{5mm}
\subsubsection{Structure of Blocks}
\vspace{5mm}

\noindent \textit{FOR EACH SIMULATION}
\lstinputlisting[style=blockdefinition]{./mf6ivar/tex/gwf-ghb-options.dat}
\lstinputlisting[style=blockdefinition]{./mf6ivar/tex/gwf-ghb-dimensions.dat}
\vspace{5mm}
\noindent \textit{FOR ANY STRESS PERIOD}
\lstinputlisting[style=blockdefinition]{./mf6ivar/tex/gwf-ghb-period.dat}
\packageperioddescription

\vspace{5mm}
\subsubsection{Explanation of Variables}
\begin{description}
\input{./mf6ivar/tex/gwf-ghb-desc.tex}
\end{description}

\vspace{5mm}
\subsubsection{Example Input File}
\lstinputlisting[style=inputfile]{./mf6ivar/examples/gwf-ghb-example.dat}

\vspace{5mm}
\subsubsection{Available observation types}
General-Head Boundary Package observations include the simulated general-head boundary flow rates (\texttt{ghb}) and the general-head boundary discharge that is available for the MVR package (\texttt{to-mvr}). The data required for each GHB Package observation type is defined in table~\ref{table:gwf-ghbobstype}. The sum of \texttt{ghb} and \texttt{to-mvr} is equal to the simulated general-head boundary flow rate. Negative and positive values for an observation represent a loss from and gain to the GWF model, respectively.

\begin{longtable}{p{2cm} p{2.75cm} p{2cm} p{1.25cm} p{7cm}}
\caption{Available GHB Package observation types} \tabularnewline

\hline
\hline
\textbf{Stress Package} & \textbf{Observation type} & \textbf{ID} & \textbf{ID2} & \textbf{Description} \\
\hline
\endhead

\hline
\endfoot

GHB & ghb & cellid or boundname & -- & Flow between the groundwater system and a general-head boundary or group of general-head boundaries. \\
GHB & to-mvr & cellid or boundname & -- & General-head boundary discharge that is available for the MVR package from a general-head boundary or group of general-head boundaries.
\label{table:gwf-ghbobstype}
\end{longtable}

\vspace{5mm}
\subsubsection{Example Observation Input File}
\lstinputlisting[style=inputfile]{./mf6ivar/examples/gwf-ghb-example-obs.dat}
